% Options for packages loaded elsewhere
\PassOptionsToPackage{unicode}{hyperref}
\PassOptionsToPackage{hyphens}{url}
%
\documentclass[
]{article}
\usepackage{amsmath,amssymb}
\usepackage{lmodern}
\usepackage{ifxetex,ifluatex}
\ifnum 0\ifxetex 1\fi\ifluatex 1\fi=0 % if pdftex
  \usepackage[T1]{fontenc}
  \usepackage[utf8]{inputenc}
  \usepackage{textcomp} % provide euro and other symbols
\else % if luatex or xetex
  \usepackage{unicode-math}
  \defaultfontfeatures{Scale=MatchLowercase}
  \defaultfontfeatures[\rmfamily]{Ligatures=TeX,Scale=1}
\fi
% Use upquote if available, for straight quotes in verbatim environments
\IfFileExists{upquote.sty}{\usepackage{upquote}}{}
\IfFileExists{microtype.sty}{% use microtype if available
  \usepackage[]{microtype}
  \UseMicrotypeSet[protrusion]{basicmath} % disable protrusion for tt fonts
}{}
\makeatletter
\@ifundefined{KOMAClassName}{% if non-KOMA class
  \IfFileExists{parskip.sty}{%
    \usepackage{parskip}
  }{% else
    \setlength{\parindent}{0pt}
    \setlength{\parskip}{6pt plus 2pt minus 1pt}}
}{% if KOMA class
  \KOMAoptions{parskip=half}}
\makeatother
\usepackage{xcolor}
\IfFileExists{xurl.sty}{\usepackage{xurl}}{} % add URL line breaks if available
\IfFileExists{bookmark.sty}{\usepackage{bookmark}}{\usepackage{hyperref}}
\hypersetup{
  pdftitle={Statystyczn analiza danych - projekt dotyczący badania nad rynkiem samochodowym},
  pdfauthor={Aldona Swirad},
  hidelinks,
  pdfcreator={LaTeX via pandoc}}
\urlstyle{same} % disable monospaced font for URLs
\usepackage[margin=1in]{geometry}
\usepackage{color}
\usepackage{fancyvrb}
\newcommand{\VerbBar}{|}
\newcommand{\VERB}{\Verb[commandchars=\\\{\}]}
\DefineVerbatimEnvironment{Highlighting}{Verbatim}{commandchars=\\\{\}}
% Add ',fontsize=\small' for more characters per line
\usepackage{framed}
\definecolor{shadecolor}{RGB}{248,248,248}
\newenvironment{Shaded}{\begin{snugshade}}{\end{snugshade}}
\newcommand{\AlertTok}[1]{\textcolor[rgb]{0.94,0.16,0.16}{#1}}
\newcommand{\AnnotationTok}[1]{\textcolor[rgb]{0.56,0.35,0.01}{\textbf{\textit{#1}}}}
\newcommand{\AttributeTok}[1]{\textcolor[rgb]{0.77,0.63,0.00}{#1}}
\newcommand{\BaseNTok}[1]{\textcolor[rgb]{0.00,0.00,0.81}{#1}}
\newcommand{\BuiltInTok}[1]{#1}
\newcommand{\CharTok}[1]{\textcolor[rgb]{0.31,0.60,0.02}{#1}}
\newcommand{\CommentTok}[1]{\textcolor[rgb]{0.56,0.35,0.01}{\textit{#1}}}
\newcommand{\CommentVarTok}[1]{\textcolor[rgb]{0.56,0.35,0.01}{\textbf{\textit{#1}}}}
\newcommand{\ConstantTok}[1]{\textcolor[rgb]{0.00,0.00,0.00}{#1}}
\newcommand{\ControlFlowTok}[1]{\textcolor[rgb]{0.13,0.29,0.53}{\textbf{#1}}}
\newcommand{\DataTypeTok}[1]{\textcolor[rgb]{0.13,0.29,0.53}{#1}}
\newcommand{\DecValTok}[1]{\textcolor[rgb]{0.00,0.00,0.81}{#1}}
\newcommand{\DocumentationTok}[1]{\textcolor[rgb]{0.56,0.35,0.01}{\textbf{\textit{#1}}}}
\newcommand{\ErrorTok}[1]{\textcolor[rgb]{0.64,0.00,0.00}{\textbf{#1}}}
\newcommand{\ExtensionTok}[1]{#1}
\newcommand{\FloatTok}[1]{\textcolor[rgb]{0.00,0.00,0.81}{#1}}
\newcommand{\FunctionTok}[1]{\textcolor[rgb]{0.00,0.00,0.00}{#1}}
\newcommand{\ImportTok}[1]{#1}
\newcommand{\InformationTok}[1]{\textcolor[rgb]{0.56,0.35,0.01}{\textbf{\textit{#1}}}}
\newcommand{\KeywordTok}[1]{\textcolor[rgb]{0.13,0.29,0.53}{\textbf{#1}}}
\newcommand{\NormalTok}[1]{#1}
\newcommand{\OperatorTok}[1]{\textcolor[rgb]{0.81,0.36,0.00}{\textbf{#1}}}
\newcommand{\OtherTok}[1]{\textcolor[rgb]{0.56,0.35,0.01}{#1}}
\newcommand{\PreprocessorTok}[1]{\textcolor[rgb]{0.56,0.35,0.01}{\textit{#1}}}
\newcommand{\RegionMarkerTok}[1]{#1}
\newcommand{\SpecialCharTok}[1]{\textcolor[rgb]{0.00,0.00,0.00}{#1}}
\newcommand{\SpecialStringTok}[1]{\textcolor[rgb]{0.31,0.60,0.02}{#1}}
\newcommand{\StringTok}[1]{\textcolor[rgb]{0.31,0.60,0.02}{#1}}
\newcommand{\VariableTok}[1]{\textcolor[rgb]{0.00,0.00,0.00}{#1}}
\newcommand{\VerbatimStringTok}[1]{\textcolor[rgb]{0.31,0.60,0.02}{#1}}
\newcommand{\WarningTok}[1]{\textcolor[rgb]{0.56,0.35,0.01}{\textbf{\textit{#1}}}}
\usepackage{graphicx}
\makeatletter
\def\maxwidth{\ifdim\Gin@nat@width>\linewidth\linewidth\else\Gin@nat@width\fi}
\def\maxheight{\ifdim\Gin@nat@height>\textheight\textheight\else\Gin@nat@height\fi}
\makeatother
% Scale images if necessary, so that they will not overflow the page
% margins by default, and it is still possible to overwrite the defaults
% using explicit options in \includegraphics[width, height, ...]{}
\setkeys{Gin}{width=\maxwidth,height=\maxheight,keepaspectratio}
% Set default figure placement to htbp
\makeatletter
\def\fps@figure{htbp}
\makeatother
\setlength{\emergencystretch}{3em} % prevent overfull lines
\providecommand{\tightlist}{%
  \setlength{\itemsep}{0pt}\setlength{\parskip}{0pt}}
\setcounter{secnumdepth}{-\maxdimen} % remove section numbering
\ifluatex
  \usepackage{selnolig}  % disable illegal ligatures
\fi

\title{Statystyczn analiza danych - projekt dotyczący badania nad
rynkiem samochodowym}
\author{Aldona Swirad}
\date{2023-06-09}

\begin{document}
\maketitle

\hypertarget{spis-treux15bci}{%
\section{Spis treści}\label{spis-treux15bci}}

\hypertarget{wstux119p}{%
\section{Wstęp}\label{wstux119p}}

\hypertarget{ux17aruxf3dux142o}{%
\paragraph{Źródło}\label{ux17aruxf3dux142o}}

Dane pochodzą z ogólnodostępnej platformy Kaggle.
\url{https://www.kaggle.com/datasets/gagandeep16/car-sales}

\hypertarget{opis-danych}{%
\paragraph{Opis danych}\label{opis-danych}}

Wybrane dane dotyczą samochodów. Możemy tam znaleźć informacje na temat
ich poszczególnyhc cech, takich jak:

\begin{itemize}
\tightlist
\item
  Manufacturer - Marka
\item
  Model - model
\item
  Sales\_in\_thousands - sprzedaż w tysiącach
\item
  \_\_year\_resale\_value - roczna wartość odsprzedaży
\item
  Vehicle\_type - rodzaj
\item
  Price\_in\_thousands - cena w tysiącach
\item
  Engine\_size - pojemność silnika
\item
  Horsepower - liczba koni mechanicznych
\item
  Wheelbase - rozstaw osi
\item
  Width - szerokokść
\item
  Length - długość
\item
  Curb\_weight - masa wlasna pojazdu
\item
  Fuel\_capacity - pojemnosc baku
\item
  Fuel\_efficiency - efektywnosc spalania
\end{itemize}

Chciałam wybrać temat, który mnie zainteresuje i pozwoli na zdobycie
praktycznych umiejętności w analizie danych. Ze względu na moje osobiste
zainteresowania oraz wymagania akademickie, analiza danych na temat
samochodów wydała mi się interesującym wyborem. Jestem ciekawa
zależności między różnymi parametrami samochodów, takimi jak moc
silnika, czy cena oraz jakie wnioski można wyciągnąć na podstawie
analizy statystycznej takich danych. Ponadto, widzę praktyczne
zastosowanie takiej analizy w przemyśle motoryzacyjnym, co również mnie
zainteresowało i skłoniło do wyboru tego tematu do mojego projektu.

\hypertarget{wczytanie-danych-i-ich-obruxf3bka}{%
\paragraph{Wczytanie danych i ich
obróbka}\label{wczytanie-danych-i-ich-obruxf3bka}}

\begin{Shaded}
\begin{Highlighting}[]
\CommentTok{\# użyte biblioteki}
\FunctionTok{library}\NormalTok{(e1071)}
\FunctionTok{library}\NormalTok{(tidyverse)}
\FunctionTok{library}\NormalTok{(hrbrthemes)}
\FunctionTok{library}\NormalTok{(viridis)}
\DocumentationTok{\#\# wczytywanie danych}
\FunctionTok{setwd}\NormalTok{(}\StringTok{\textquotesingle{}D:/IiAD sem\_4/Statystyka/Projekt\_SAD\textquotesingle{}}\NormalTok{)}

\NormalTok{cars }\OtherTok{\textless{}{-}} \FunctionTok{read.csv}\NormalTok{(}\StringTok{\textquotesingle{}Car\_sales.csv\textquotesingle{}}\NormalTok{, }\AttributeTok{header =}\NormalTok{ T, }\AttributeTok{sep =} \StringTok{\textquotesingle{},\textquotesingle{}}\NormalTok{)}

\NormalTok{cars }\OtherTok{\textless{}{-}} \FunctionTok{na.omit}\NormalTok{(cars)}
\FunctionTok{colnames}\NormalTok{(cars)[}\DecValTok{3}\NormalTok{] }\OtherTok{\textless{}{-}} \StringTok{"Sales"}
\FunctionTok{colnames}\NormalTok{(cars)[}\DecValTok{6}\NormalTok{] }\OtherTok{\textless{}{-}} \StringTok{"Price"}
\NormalTok{cars}\SpecialCharTok{$}\NormalTok{Sales }\OtherTok{\textless{}{-}}\NormalTok{ cars}\SpecialCharTok{$}\NormalTok{Sales}\SpecialCharTok{*}\DecValTok{1000}
\NormalTok{cars}\SpecialCharTok{$}\NormalTok{Price }\OtherTok{\textless{}{-}}\NormalTok{ cars}\SpecialCharTok{$}\NormalTok{Price}\SpecialCharTok{*}\DecValTok{1000}
\NormalTok{cars }\OtherTok{\textless{}{-}}\NormalTok{ cars[, }\SpecialCharTok{{-}}\FunctionTok{c}\NormalTok{(}\DecValTok{15}\NormalTok{,}\DecValTok{16}\NormalTok{)]}
\end{Highlighting}
\end{Shaded}

\hypertarget{opis-wybranych-bibliotek}{%
\paragraph{Opis wybranych bibliotek}\label{opis-wybranych-bibliotek}}

\begin{enumerate}
\def\labelenumi{\arabic{enumi}.}
\item
  library(e1071): Biblioteka ``e1071'' jest jednym z najważniejszych
  pakietów w języku R dla analizy danych i uczenia maszynowego. Zapewnia
  różnorodne narzędzia i funkcje do klasyfikacji, regresji, analizy
  skupień i wiele innych technik statystycznych. W szczególności,
  biblioteka e1071 dostarcza implementacje popularnych algorytmów
  uczenia maszynowego, takich jak maszyny wektorów nośnych (SVM), metody
  klasyfikacji Bayesa i wiele innych. Jest to niezwykle przydatne
  narzędzie dla badaczy i analityków danych, którzy chcą wykorzystać
  zaawansowane techniki analizy danych w języku R.
\item
  library(tidyverse): Biblioteka ``tidyverse'' to zestaw kilku
  powiązanych pakietów R, które są wykorzystywane do manipulacji,
  wizualizacji i analizy danych. W skład tidyverse wchodzą popularne
  pakiety, takie jak ggplot2, dplyr, tidyr i wiele innych. Dzięki
  tidyverse możesz łatwo przeprowadzać zaawansowane operacje na danych,
  takie jak filtrowanie, sortowanie, grupowanie, łączenie danych z
  różnych źródeł itp. Biblioteka ta jest ceniona za spójność i
  czytelność kodu, co ułatwia pracę z danymi w R.
\item
  library(hrbrthemes): Biblioteka ``hrbrthemes'' to pakiet R zawierający
  zestaw tematów graficznych (themes) do wykorzystania w pakiecie
  ggplot2. Dostarcza wiele estetycznych i profesjonalnie wyglądających
  tematów, które można stosować do tworzenia wykresów o wysokiej
  jakości. Hrbrthemes oferuje różnorodne opcje kolorystyczne, układy i
  style czcionek, które mogą być dostosowane do konkretnych potrzeb
  wizualizacji danych. Jest to przydatne narzędzie dla osób, które chcą
  poprawić estetykę swoich wykresów i prezentacji danych.
\item
  library(viridis): Biblioteka ``viridis'' to pakiet R dostarczający
  zestaw wysokiej jakości kolorowych map gradientowych, które są
  szczególnie przydatne przy tworzeniu wizualizacji danych. Kolorowe
  mapy gradientowe viridis są zaprojektowane tak, aby były czytelne
  również dla osób z deficytami wzroku, dzięki czemu są popularne w
  dziedzinie wizualizacji naukowych i statystycznych. Biblioteka viridis
  oferuje różne palety kolorów, które można łatwo stosować w pakiecie
  ggplot2 lub innych narzędziach do wizualizacji danych w R.
\end{enumerate}

\hypertarget{charakterystyki-liczbowe}{%
\paragraph{Charakterystyki liczbowe}\label{charakterystyki-liczbowe}}

\hypertarget{srednia-cena-wzgledem-marki}{%
\subparagraph{Srednia cena wzgledem
marki}\label{srednia-cena-wzgledem-marki}}

\begin{Shaded}
\begin{Highlighting}[]
\CommentTok{\# obliczanie sredniej}
\CommentTok{\#srednia wzgledem marki (grupowanie)}
\NormalTok{vprices }\OtherTok{\textless{}{-}} \FunctionTok{c}\NormalTok{()}
\NormalTok{vcounter }\OtherTok{\textless{}{-}} \FunctionTok{c}\NormalTok{()}
\NormalTok{price }\OtherTok{\textless{}{-}}\NormalTok{ cars}\SpecialCharTok{$}\NormalTok{Price[}\DecValTok{1}\NormalTok{]}
\NormalTok{counter }\OtherTok{\textless{}{-}} \DecValTok{0}



\ControlFlowTok{for}\NormalTok{ (i }\ControlFlowTok{in} \DecValTok{2}\SpecialCharTok{:}\FunctionTok{length}\NormalTok{(cars}\SpecialCharTok{$}\NormalTok{Price))}
  \ControlFlowTok{if}\NormalTok{ (cars}\SpecialCharTok{$}\NormalTok{Manufacturer[i}\DecValTok{{-}1}\NormalTok{]}\SpecialCharTok{==}\NormalTok{cars}\SpecialCharTok{$}\NormalTok{Manufacturer[i])\{}
\NormalTok{    price }\OtherTok{\textless{}{-}}\NormalTok{ price }\SpecialCharTok{+}\NormalTok{ cars}\SpecialCharTok{$}\NormalTok{Price[i]}
\NormalTok{    counter }\OtherTok{\textless{}{-}}\NormalTok{ counter }\SpecialCharTok{+} \DecValTok{1}
    \ControlFlowTok{if}\NormalTok{(i}\SpecialCharTok{==}\FunctionTok{length}\NormalTok{(cars}\SpecialCharTok{$}\NormalTok{Price))\{}
\NormalTok{      price }\OtherTok{\textless{}{-}}\NormalTok{ price }\SpecialCharTok{+}\NormalTok{ cars}\SpecialCharTok{$}\NormalTok{Price[i]}
\NormalTok{      counter }\OtherTok{\textless{}{-}}\NormalTok{ counter }\SpecialCharTok{+} \DecValTok{1}
\NormalTok{      vprices }\OtherTok{\textless{}{-}} \FunctionTok{append}\NormalTok{(vprices, price)}
\NormalTok{      vcounter }\OtherTok{\textless{}{-}} \FunctionTok{append}\NormalTok{(vcounter, counter)}
\NormalTok{    \}}
\NormalTok{  \}}\ControlFlowTok{else}\NormalTok{\{}
\NormalTok{    price }\OtherTok{\textless{}{-}}\NormalTok{ price }\SpecialCharTok{+}\NormalTok{ cars}\SpecialCharTok{$}\NormalTok{Price[i]}
\NormalTok{    counter }\OtherTok{\textless{}{-}}\NormalTok{ counter }\SpecialCharTok{+} \DecValTok{1}
\NormalTok{    vprices }\OtherTok{\textless{}{-}} \FunctionTok{append}\NormalTok{(vprices, price)}
\NormalTok{    vcounter }\OtherTok{\textless{}{-}} \FunctionTok{append}\NormalTok{(vcounter, counter)}
\NormalTok{    counter }\OtherTok{\textless{}{-}} \DecValTok{0}
\NormalTok{    price }\OtherTok{\textless{}{-}} \DecValTok{0}
\NormalTok{  \}}

\CommentTok{\#srednie}
\NormalTok{avg\_by\_manu }\OtherTok{\textless{}{-}}\NormalTok{ vprices}\SpecialCharTok{/}\NormalTok{vcounter}
\end{Highlighting}
\end{Shaded}

\hypertarget{wykres-sux142upkowy-ux15brednich-cen}{%
\paragraph{Wykres słupkowy średnich
cen}\label{wykres-sux142upkowy-ux15brednich-cen}}

\includegraphics{Cars_project_files/figure-latex/unnamed-chunk-3-1.pdf}

\hypertarget{odchylenie-standardowe-ux15brednich-cen-wzglux119dem-marek}{%
\paragraph{Odchylenie standardowe średnich cen względem
marek}\label{odchylenie-standardowe-ux15brednich-cen-wzglux119dem-marek}}

\begin{Shaded}
\begin{Highlighting}[]
\CommentTok{\# odchylenie}
\NormalTok{stand\_dev }\OtherTok{\textless{}{-}} \FunctionTok{sd}\NormalTok{(avg\_by\_manu)}
\end{Highlighting}
\end{Shaded}

\begin{verbatim}
## [1] 10658.29
\end{verbatim}

\hypertarget{dominanta-dla-rozmiaruxf3w-silnikuxf3w}{%
\paragraph{Dominanta dla rozmiarów
silników}\label{dominanta-dla-rozmiaruxf3w-silnikuxf3w}}

\begin{Shaded}
\begin{Highlighting}[]
\CommentTok{\# dominanta}
\NormalTok{getmode }\OtherTok{\textless{}{-}} \ControlFlowTok{function}\NormalTok{(v) \{}
\NormalTok{   uniqv }\OtherTok{\textless{}{-}} \FunctionTok{unique}\NormalTok{(v)}
\NormalTok{   uniqv[}\FunctionTok{which.max}\NormalTok{(}\FunctionTok{tabulate}\NormalTok{(}\FunctionTok{match}\NormalTok{(v, uniqv)))]}
\NormalTok{\}}
\FunctionTok{getmode}\NormalTok{(cars}\SpecialCharTok{$}\NormalTok{Engine\_size)}
\end{Highlighting}
\end{Shaded}

\begin{verbatim}
## [1] 2
\end{verbatim}

\hypertarget{kwantyle-wux15bruxf3d-rozmiaruxf3w-silnikuxf3w}{%
\paragraph{Kwantyle wśród rozmiarów
silników}\label{kwantyle-wux15bruxf3d-rozmiaruxf3w-silnikuxf3w}}

\begin{Shaded}
\begin{Highlighting}[]
\CommentTok{\# kwantyle}
\NormalTok{kwantyle }\OtherTok{\textless{}{-}} \FunctionTok{quantile}\NormalTok{(cars}\SpecialCharTok{$}\NormalTok{Engine\_size)}
\end{Highlighting}
\end{Shaded}

\begin{verbatim}
##   0%  25%  50%  75% 100% 
##  1.0  2.2  3.0  3.8  8.0
\end{verbatim}

\hypertarget{korelacja-miux119dzy-wielkoux15bciux105-silnika-a-jego-mocux105}{%
\paragraph{Korelacja między wielkością silnika, a jego
mocą}\label{korelacja-miux119dzy-wielkoux15bciux105-silnika-a-jego-mocux105}}

\begin{Shaded}
\begin{Highlighting}[]
\CommentTok{\#korelacja}
\FunctionTok{plot}\NormalTok{(cars}\SpecialCharTok{$}\NormalTok{Engine\_size, cars}\SpecialCharTok{$}\NormalTok{Horsepower, }\AttributeTok{xlab =} \StringTok{\textquotesingle{}Rozmiar silnika\textquotesingle{}}\NormalTok{, }\AttributeTok{ylab =} \StringTok{\textquotesingle{}Konie mechaniczne\textquotesingle{}}\NormalTok{)}
\end{Highlighting}
\end{Shaded}

\includegraphics{Cars_project_files/figure-latex/unnamed-chunk-9-1.pdf}

\begin{verbatim}
## [1] 0.8616183
\end{verbatim}

\hypertarget{rozkux142ad-wartoux15bci-mocy-silnika}{%
\paragraph{Rozkład wartości mocy
silnika}\label{rozkux142ad-wartoux15bci-mocy-silnika}}

\begin{Shaded}
\begin{Highlighting}[]
\CommentTok{\#rozklad}
\NormalTok{rozklad\_hp }\OtherTok{\textless{}{-}} \FunctionTok{plot}\NormalTok{(}\FunctionTok{density}\NormalTok{(cars}\SpecialCharTok{$}\NormalTok{Horsepower), }\AttributeTok{main =} \StringTok{\textquotesingle{}Rozkład wartości mocy silnika\textquotesingle{}}\NormalTok{, }\AttributeTok{ylab =} \StringTok{\textquotesingle{}ilość samochodów o danej mocy\textquotesingle{}}\NormalTok{, }\AttributeTok{xlab =} \StringTok{\textquotesingle{}moc silnika\textquotesingle{}}\NormalTok{)}
\end{Highlighting}
\end{Shaded}

\includegraphics{Cars_project_files/figure-latex/unnamed-chunk-11-1.pdf}

\begin{quote}
Współczynnik zmienności
\end{quote}

\begin{Shaded}
\begin{Highlighting}[]
\CommentTok{\#wspolczynnik zmiennosci}
\NormalTok{cv }\OtherTok{\textless{}{-}} \FunctionTok{sd}\NormalTok{(cars}\SpecialCharTok{$}\NormalTok{Horsepower)}\SpecialCharTok{/}\FunctionTok{mean}\NormalTok{(cars}\SpecialCharTok{$}\NormalTok{Horsepower)}
\end{Highlighting}
\end{Shaded}

\begin{verbatim}
## [1] 0.3232079
\end{verbatim}

\begin{quote}
Współczynnik asymetrii
\end{quote}

\begin{Shaded}
\begin{Highlighting}[]
\NormalTok{as }\OtherTok{\textless{}{-}} \DecValTok{3}\SpecialCharTok{*}\NormalTok{(}\FunctionTok{mean}\NormalTok{(cars}\SpecialCharTok{$}\NormalTok{Horsepower)}\SpecialCharTok{{-}}\FunctionTok{median}\NormalTok{(cars}\SpecialCharTok{$}\NormalTok{Horsepower))}\SpecialCharTok{/}\FunctionTok{sd}\NormalTok{(cars}\SpecialCharTok{$}\NormalTok{Horsepower)}
\end{Highlighting}
\end{Shaded}

\begin{verbatim}
## [1] 0.3216518
\end{verbatim}

\begin{quote}
Współczynnik splaszczenia
\end{quote}

\begin{Shaded}
\begin{Highlighting}[]
\NormalTok{kurtoza }\OtherTok{\textless{}{-}} \FunctionTok{moment}\NormalTok{(cars}\SpecialCharTok{$}\NormalTok{Horsepower, }\AttributeTok{order=}\DecValTok{4}\NormalTok{, }\AttributeTok{center=}\ConstantTok{TRUE}\NormalTok{)}\SpecialCharTok{/}\NormalTok{(}\FunctionTok{sd}\NormalTok{(cars}\SpecialCharTok{$}\NormalTok{Horsepower)}\SpecialCharTok{\^{}}\DecValTok{3}\NormalTok{)}
\end{Highlighting}
\end{Shaded}

\begin{verbatim}
## [1] 345.8519
\end{verbatim}

\hypertarget{wykres-pudeux142kowy-efektywnoux15bci-spalania}{%
\paragraph{Wykres pudełkowy efektywności
spalania}\label{wykres-pudeux142kowy-efektywnoux15bci-spalania}}

\begin{Shaded}
\begin{Highlighting}[]
\CommentTok{\#wykres pudełkowy}

\FunctionTok{ggplot}\NormalTok{(cars, }\FunctionTok{aes}\NormalTok{(}\AttributeTok{x=}\FunctionTok{as.factor}\NormalTok{(cars}\SpecialCharTok{$}\NormalTok{Vehicle\_type), }\AttributeTok{y=}\NormalTok{cars}\SpecialCharTok{$}\NormalTok{Fuel\_efficiency), ) }\SpecialCharTok{+} 
    \FunctionTok{geom\_boxplot}\NormalTok{(}\AttributeTok{fill=}\StringTok{"slateblue"}\NormalTok{, }\AttributeTok{alpha=}\FloatTok{0.2}\NormalTok{) }\SpecialCharTok{+}
    \FunctionTok{xlab}\NormalTok{(}\StringTok{"typ samochodu"}\NormalTok{)}\SpecialCharTok{+}
\FunctionTok{ylab}\NormalTok{(}\StringTok{\textquotesingle{}efektywość spalania\textquotesingle{}}\NormalTok{)}
\end{Highlighting}
\end{Shaded}

\includegraphics{Cars_project_files/figure-latex/unnamed-chunk-18-1.pdf}

\hypertarget{hipotezy-statystyczne}{%
\subsection{Hipotezy statystyczne}\label{hipotezy-statystyczne}}

\hypertarget{ux15brednia-cena-samochodu-to-60-000}{%
\subsubsection{Średnia cena samochodu to 60
000}\label{ux15brednia-cena-samochodu-to-60-000}}

\begin{itemize}
\tightlist
\item
  H0-Średnia cena samochodu to 60 000
\item
  H1-Średnia cena samochodu nie jest równa 60 000
\item
  poziom istotności \(\alpha\) - 0,1
\end{itemize}

\begin{Shaded}
\begin{Highlighting}[]
\FunctionTok{t.test}\NormalTok{(cars}\SpecialCharTok{$}\NormalTok{Sales, }\AttributeTok{mu=}\DecValTok{60000}\NormalTok{, }\AttributeTok{alternalive=}\StringTok{"less"}\NormalTok{, }\AttributeTok{conf.level =} \FloatTok{0.1}\NormalTok{)}
\end{Highlighting}
\end{Shaded}

\begin{verbatim}
## 
##  One Sample t-test
## 
## data:  cars$Sales
## t = -0.12792, df = 116, p-value = 0.8984
## alternative hypothesis: true mean is not equal to 60000
## 10 percent confidence interval:
##  58238.42 59986.22
## sample estimates:
## mean of x 
##  59112.32
\end{verbatim}

\begin{Shaded}
\begin{Highlighting}[]
\CommentTok{\#Średnia cena samochodu}
\FunctionTok{mean}\NormalTok{(cars}\SpecialCharTok{$}\NormalTok{Sales)}
\end{Highlighting}
\end{Shaded}

\begin{verbatim}
## [1] 59112.32
\end{verbatim}

Wartość funkcji testującej należy do obszaru krytycznego, więc H0
odrzucamy. Średnia samochodu dla populacji nie wynosi 60000.

\hypertarget{cena-samochoduxf3w-ma-rozkux142ad-normalny}{%
\subsubsection{Cena samochodów ma rozkład
normalny}\label{cena-samochoduxf3w-ma-rozkux142ad-normalny}}

\begin{itemize}
\tightlist
\item
  H0-Cena samochodów ma rozkład normalny
\item
  H1-Cena samochodów nie ma rozkładu normalnego
\item
  poziom istotności \(\alpha\) - 0,1
\end{itemize}

\begin{Shaded}
\begin{Highlighting}[]
\FunctionTok{shapiro.test}\NormalTok{(cars}\SpecialCharTok{$}\NormalTok{Sales)}
\end{Highlighting}
\end{Shaded}

\begin{verbatim}
## 
##  Shapiro-Wilk normality test
## 
## data:  cars$Sales
## W = 0.67779, p-value = 1.123e-14
\end{verbatim}

Wartość p \textless{} 0.05, więc H0 odrzucamy. Cena samochódów nie ma
rozkładu noramalnego.

\begin{Shaded}
\begin{Highlighting}[]
\FunctionTok{plot}\NormalTok{(}\FunctionTok{density}\NormalTok{(cars}\SpecialCharTok{$}\NormalTok{Sales), }\AttributeTok{ylab =} \StringTok{\textquotesingle{}\textquotesingle{}}\NormalTok{, }\AttributeTok{xlab =} \StringTok{\textquotesingle{} \textquotesingle{}}\NormalTok{, }\AttributeTok{main=}\StringTok{\textquotesingle{}\textquotesingle{}}\NormalTok{)}
\end{Highlighting}
\end{Shaded}

\includegraphics{Cars_project_files/figure-latex/unnamed-chunk-22-1.pdf}

\hypertarget{rozkux142ad-pojemnoux15bci-silnika-i-mocy-samochodu-jest-podobny}{%
\subsubsection{Rozkład pojemności silnika i mocy samochodu jest
podobny}\label{rozkux142ad-pojemnoux15bci-silnika-i-mocy-samochodu-jest-podobny}}

\begin{Shaded}
\begin{Highlighting}[]
\FunctionTok{ks.test}\NormalTok{(cars}\SpecialCharTok{$}\NormalTok{Engine\_size,cars}\SpecialCharTok{$}\NormalTok{Horsepower)}
\end{Highlighting}
\end{Shaded}

\begin{verbatim}
## Warning in ks.test.default(cars$Engine_size, cars$Horsepower): wartość
## prawdopodobieństwa w obecności powtórzonych wartości będzie przybliżona
\end{verbatim}

\begin{verbatim}
## 
##  Asymptotic two-sample Kolmogorov-Smirnov test
## 
## data:  cars$Engine_size and cars$Horsepower
## D = 1, p-value < 2.2e-16
## alternative hypothesis: two-sided
\end{verbatim}

\hypertarget{section}{%
\subsubsection{}\label{section}}

\hypertarget{wnioski}{%
\subsubsection{Wnioski}\label{wnioski}}

Dzięki szczegółowej analizie jesteśmy w stanie poznać badany przez nas
temat, np:rynek samochodowy oraz wyciągać wnioski i odszukać nawet
nieintuicyjne fakty. Projekt pomógł mi w głębszym zaznajomieniu się z
używaniem narzędzi statystycznych. Poszerzyłam także swoją znajomość
języka R oraz środowiska RStudio. Powyższy projekt jest podstawą do
dalszego zagłębiania się w statystykę od jej strony prkatycznej, a
poznane umiejętności mogą okazać się bardzo przydatne w przyszłej pracy.

\end{document}
